\section{Technische Dokumentation}

\subsection{Einleitung}

Die generelle Zielsetzung sah die Erstellung eines Tools vor, das auf Grundlage von Benutzerpräferenzen und historischen Daten eines Nutzers personalisierte Reisevorschläge erstellt. Dabei solle es zur Kalkulierung Echtzeitinformationen zu Flug und Zugverbindungen oder Events berücksichtigen. 
Die technischen Anforderungen dabei bestanden in der Nutzung von APIs, einem Backend für Benutzer- und Datenmanagement, sowie einem Data-Science Algorithmus für die Personalisierung und Empfehlung.  
Zusätzlich sollte – nach Möglichkeit – das Tool durch eine anschließende Datenanalyse, bzw. eine Datenvisualisierung erweitert werden.  
Es ist jedoch anzumerken, dass im Laufe der Entwicklung die Anforderungen reduziert wurden mit dem Fokus auf die Struktur und ein erkennbares funktionierendes System.

\subsection{Architekturübersicht}

Im Projekt wurde entschieden, eine Microservice-basierte Webanwendung zu konzipieren. Die Gesamtarchitektur gliedert sich in ein Frontend, mehrere spezialisierte Backend-Services sowie eine Datenbank. Zusätzlich werden externe APIs zur Datenanreicherung eingebunden.  
Das Frontend übernimmt die Registrierung neuer Nutzer, die Verwaltung von Nutzerprofilen sowie die Abfrage personalisierter Reiseempfehlungen. Es wurde mit dem Angular-Framework als Single-Page-Application (SPA) realisiert.  

Das Backend ist in zwei separate Microservices unterteilt:
\begin{itemize}
  \item \textbf{auth\_service}: übernimmt die Authentifizierung und Verwaltung von Nutzerkonten und Präferenzen
  \item \textbf{travel\_service}: verarbeitet Nutzeranfragen, verwaltet geplante Reisen und generiert personalisierte Empfehlungen
\end{itemize}

Beide Microservices greifen auf eine MongoDB-Datenbank zu, in der Nutzerstammdaten sowie individuelle Anfragen und Reiseinformationen gespeichert werden.  
Abschließend wurden sämtliche Anwendungen containerisiert und über ein Kubernetes-Cluster orchestriert. Dies ermöglicht eine standardisierte, skalierbare und portable Bereitstellung der Anwendung.

\section*{1.3 Konzepte aus der Vorlesung}

\subsection*{1.3.1 Webentwicklung}

\subsubsection*{1.3.1.1 Single-Page Application (SPA) mit Angular}

Für die Entwicklung des Frontends wurde das Angular-Framework eingesetzt. Die Anwendung basiert auf einer komponentenbasierten Architektur, bei der jede Funktionalität – wie Login, Registrierung oder Ergebnisanzeige – in eigenständigen Bausteinen gekapselt ist. Zur Gestaltung der Benutzeroberfläche kam Angular Material mit vorgefertigten UI-Komponenten wie Buttons, Formularfeldern und Navigationsleisten zum Einsatz.

\textbf{Verwendete Konzepte aus der Vorlesung:}
\begin{itemize}
  \item HTML- und CSS-Grundlagen für Struktur und Gestaltung
  \item Komponentenarchitektur zur modularen Entwicklung
  \item Angular-Routing für seitenübergreifende Navigation ohne Neuladen
\end{itemize}

\subsubsection*{1.3.1.2 REST-Kommunikation}
\textit{(Kommt)}

\subsection*{1.3.2 Verteilte Systeme}

\subsubsection*{Microservice-Architektur}

Das Backend wurde mithilfe von FastAPI in zwei Microservices umgesetzt. Die Architektur folgt dem Prinzip der Modularisierung und entspricht dem in der Vorlesung behandelten Aufbau verteilter Systeme.

\textbf{Verwendete Konzepte aus der Vorlesung:}
\begin{itemize}
  \item Aufteilung in unabhängige, lose gekoppelte Dienste
  \item Kommunikation über REST-Schnittstellen (HTTP, JSON)
\end{itemize}

\subsubsection*{Containerisierung mit Docker}

Beide Microservices wurden containerisiert und in einem Docker-Setup betrieben. Aufbau, Build und Deployment der Container orientieren sich an den in der Vorlesung vermittelten Grundlagen.

\textbf{Verwendete Konzepte aus der Vorlesung:}
\begin{itemize}
  \item Containerisierung und Virtualisierung mit Docker
  \item Aufbau eines Dockerfile zur Abbildung der Laufzeitumgebung
  \item Image-Build, Tagging und Ausführung mit \texttt{docker build} und \texttt{docker run}
\end{itemize}

\subsubsection*{Kubernetes Cluster}

Die containerisierten Microservices wurden in einem Kubernetes-Cluster orchestriert. Damit wurde ein praxisnaher Anwendungsfall der in der Vorlesung behandelten Themen zu Skalierung und Verwaltung verteilter Systeme umgesetzt.

\textbf{Verwendete Konzepte aus der Vorlesung:}
\begin{itemize}
  \item Orchestrierung von Containern
  \item Deployment und Verwaltung über Kubernetes-Ressourcen (z.\,B. Pods, Services)
\end{itemize}


\subsection{Technische Umsetzung}

\subsection{Kompromisse und Abweichungen}
